\documentclass{amsart}
\usepackage{tikz}
\usepackage{fullpage}

\definecolor{olivegreen}{cmyk}{0.64,0,0.95,0.40}
\definecolor{rawsienna}{cmyk}{0,0.72,1,0.45}

\newcommand{\GREENYELLOW}[1]{{\color{greenyellow}#1}}
\newcommand{\YELLOW}[1]{{\color{yellow}#1}}
\newcommand{\YLW}[1]{{\color{yellow}#1}}
\newcommand{\GOLDENROD}[1]{{\color{goldenrod}#1}}
\newcommand{\DANDELION}[1]{{\color{dandelion}#1}}
\newcommand{\APRICOT}[1]{{\color{apricot}#1}}
\newcommand{\PEACH}[1]{{\color{peach}#1}}
\newcommand{\MELON}[1]{{\color{melon}#1}}
\newcommand{\YELLOWORANGE}[1]{{\color{yelloworange}#1}}
\newcommand{\ORANGE}[1]{{\color{orange}#1}}
\newcommand{\BURNTORANGE}[1]{{\color{burntorange}#1}}
\newcommand{\BITTERSWEET}[1]{{\color{bittersweet}#1}}
\newcommand{\REDORANGE}[1]{{\color{redorange}#1}}
\newcommand{\MAHOGANY}[1]{{\color{mahogany}#1}}
\newcommand{\MAROON}[1]{{\color{maroon}#1}}
\newcommand{\BRICKRED}[1]{{\color{brickred}#1}}
\newcommand{\RED}[1]{{\color{red}#1}}
\newcommand{\ORANGERED}[1]{{\color{orangered}#1}}
\newcommand{\RUBINERED}[1]{{\color{rubinered}#1}}
\newcommand{\WILDSTRAWBERRY}[1]{{\color{wildstrawberry}#1}}
\newcommand{\SALMON}[1]{{\color{salmon}#1}}
\newcommand{\CARNATIONPINK}[1]{{\color{carnationpink}#1}}
\newcommand{\MAGENTA}[1]{{\color{magenta}#1}}
\newcommand{\VIOLETRED}[1]{{\color{violetred}#1}}
\newcommand{\RHODAMINE}[1]{{\color{rhodamine}#1}}
\newcommand{\MULBERRY}[1]{{\color{mulberry}#1}}
\newcommand{\REDVIOLET}[1]{{\color{redviolet}#1}}
\newcommand{\FUCHSIA}[1]{{\color{fuchsia}#1}}
\newcommand{\LAVENDER}[1]{{\color{lavender}#1}}
\newcommand{\THISTLE}[1]{{\color{thistle}#1}}
\newcommand{\ORCHID}[1]{{\color{orchid}#1}}
\newcommand{\DARKORCHID}[1]{{\color{darkorchid}#1}}
\newcommand{\PURPLE}[1]{{\color{purple}#1}}
\newcommand{\PLUM}[1]{{\color{plum}#1}}
\newcommand{\VIOLET}[1]{{\color{violet}#1}}
\newcommand{\ROYALPURPLE}[1]{{\color{royalpurple}#1}}
\newcommand{\BLUEVIOLET}[1]{{\color{blueviolet}#1}}
\newcommand{\PERIWINKLE}[1]{{\color{periwinkle}#1}}
\newcommand{\CADETBLUE}[1]{{\color{cadetblue}#1}}
\newcommand{\CORNFLOWERBLUE}[1]{{\color{cornflowerblue}#1}}
\newcommand{\MIDNIGHTBLUE}[1]{{\color{midnightblue}#1}}
\newcommand{\NAVYBLUE}[1]{{\color{navyblue}#1}}
\newcommand{\ROYALBLUE}[1]{{\color{royalblue}#1}}
\newcommand{\BLUE}[1]{{\color{blue}#1}}
\newcommand{\CERULEAN}[1]{{\color{cerulean}#1}}
\newcommand{\CYAN}[1]{{\color{cyan}#1}}
\newcommand{\PROCESSBLUE}[1]{{\color{processblue}#1}}
\newcommand{\SKYBLUE}[1]{{\color{skyblue}#1}}
\newcommand{\TURQUOISE}[1]{{\color{turquoise}#1}}
\newcommand{\TEALBLUE}[1]{{\color{tealblue}#1}}
\newcommand{\AQUAMARINE}[1]{{\color{aquamarine}#1}}
\newcommand{\BLUEGREEN}[1]{{\color{bluegreen}#1}}
\newcommand{\EMERALD}[1]{{\color{emerald}#1}}
\newcommand{\JUNGLEGREEN}[1]{{\color{junglegreen}#1}}
\newcommand{\SEAGREEN}[1]{{\color{seagreen}#1}}
\newcommand{\GREEN}[1]{{\color{green}#1}}
\newcommand{\FORESTGREEN}[1]{{\color{forestgreen}#1}}
\newcommand{\PINEGREEN}[1]{{\color{pinegreen}#1}}
\newcommand{\LIMEGREEN}[1]{{\color{limegreen}#1}}
\newcommand{\YELLOWGREEN}[1]{{\color{yellowgreen}#1}}
\newcommand{\SPRINGGREEN}[1]{{\color{springgreen}#1}}
\newcommand{\OLIVEGREEN}[1]{{\color{olivegreen}#1}}
\newcommand{\OLG}[1]{{\color{olivegreen}#1}}
\newcommand{\RAWSIENNA}[1]{{\color{rawsienna}#1}}
\newcommand{\SEPIA}[1]{{\color{sepia}#1}}
\newcommand{\BROWN}[1]{{\color{brown}#1}}
\newcommand{\TAN}[1]{{\color{tan}#1}}
\newcommand{\GRAY}[1]{{\color{gray}#1}}
\newcommand{\LGRAY}[1]{{\color{gray!40}#1}}
\newcommand{\WHITE}[1]{{\color{white}#1}}
\newcommand{\BLACK}[1]{{\color{black}#1}}


\newcommand{\bbm}       {\left[\begin{matrix}}
\newcommand{\ebm}       {\end{matrix}\right]}
\newcommand{\bsm}       {\left[\begin{smallmatrix}}
\newcommand{\esm}       {\end{smallmatrix}\right]}
\newcommand{\bpm}       {\begin{pmatrix}}
\newcommand{\epm}       {\end{pmatrix}}
\newcommand{\bcf}[2]{\left(\begin{array}{c}{#1}\\{#2}\end{array}\right)}


\newcommand{\csch}     {\operatorname{csch}}
\newcommand{\sech}     {\operatorname{sech}}
\newcommand{\arcsinh}  {\operatorname{arcsinh}}
\newcommand{\arccosh}  {\operatorname{arccosh}}
\newcommand{\arctanh}  {\operatorname{arctanh}}

\newcommand{\range}     {\operatorname{range}}
\newcommand{\trans}     {\operatorname{trans}}
\newcommand{\trc}       {\operatorname{trace}}
\newcommand{\adj}       {\operatorname{adj}}

\newcommand{\dv}        {\operatorname{div}}
\newcommand{\grad}      {\operatorname{grad}}
\newcommand{\curl}      {\operatorname{curl}}

\newcommand{\tint}{\textstyle\int}
\newcommand{\tm}{\times}
\newcommand{\sse}{\subseteq}
\newcommand{\st}{\;|\;}
\newcommand{\sm}{\setminus}
\newcommand{\iffa}      {\Leftrightarrow}
\newcommand{\xra}{\xrightarrow}

\newcommand{\half}{\tfrac{1}{2}}

\renewcommand{\:}{\colon}

\newcommand{\N}         {{\mathbb{N}}}
\newcommand{\Z}         {{\mathbb{Z}}}
\newcommand{\Q}         {{\mathbb{Q}}}
\newcommand{\R}         {{\mathbb{R}}}
\newcommand{\C}         {{\mathbb{C}}}

\newcommand{\va}        {\mathbf{a}}
\newcommand{\vb}        {\mathbf{b}}
\newcommand{\vc}        {\mathbf{c}}
\newcommand{\vd}        {\mathbf{d}}
\newcommand{\ve}        {\mathbf{e}}
\newcommand{\vf}        {\mathbf{f}}
\newcommand{\vg}        {\mathbf{g}}
\newcommand{\vh}        {\mathbf{h}}
\newcommand{\vi}        {\mathbf{i}}
\newcommand{\vj}        {\mathbf{j}}
\newcommand{\vk}        {\mathbf{k}}
\newcommand{\vl}        {\mathbf{l}}
\newcommand{\vm}        {\mathbf{m}}
\newcommand{\vn}        {\mathbf{n}}
\newcommand{\vo}        {\mathbf{o}}
\newcommand{\vp}        {\mathbf{p}}
\newcommand{\vq}        {\mathbf{q}}
\newcommand{\vr}        {\mathbf{r}}
\newcommand{\vs}        {\mathbf{s}}
\newcommand{\vt}        {\mathbf{t}}
\newcommand{\vu}        {\mathbf{u}}
\newcommand{\vv}        {\mathbf{v}}
\newcommand{\vw}        {\mathbf{w}}
\newcommand{\vx}        {\mathbf{x}}
\newcommand{\vy}        {\mathbf{y}}
\newcommand{\vz}        {\mathbf{z}}

\newcommand{\hva}       {\widehat{\mathbf{a}}}
\newcommand{\hvb}       {\widehat{\mathbf{b}}}
\newcommand{\hvc}       {\widehat{\mathbf{c}}}
\newcommand{\hvd}       {\widehat{\mathbf{d}}}
\newcommand{\hve}       {\widehat{\mathbf{e}}}
\newcommand{\hvf}       {\widehat{\mathbf{f}}}
\newcommand{\hvg}       {\widehat{\mathbf{g}}}
\newcommand{\hvh}       {\widehat{\mathbf{h}}}
\newcommand{\hvi}       {\widehat{\mathbf{i}}}
\newcommand{\hvj}       {\widehat{\mathbf{j}}}
\newcommand{\hvk}       {\widehat{\mathbf{k}}}
\newcommand{\hvl}       {\widehat{\mathbf{l}}}
\newcommand{\hvm}       {\widehat{\mathbf{m}}}
\newcommand{\hvn}       {\widehat{\mathbf{n}}}
\newcommand{\hvo}       {\widehat{\mathbf{o}}}
\newcommand{\hvp}       {\widehat{\mathbf{p}}}
\newcommand{\hvq}       {\widehat{\mathbf{q}}}
\newcommand{\hvr}       {\widehat{\mathbf{r}}}
\newcommand{\hvs}       {\widehat{\mathbf{s}}}
\newcommand{\hvt}       {\widehat{\mathbf{t}}}
\newcommand{\hvu}       {\widehat{\mathbf{u}}}
\newcommand{\hvv}       {\widehat{\mathbf{v}}}
\newcommand{\hvw}       {\widehat{\mathbf{w}}}
\newcommand{\hvx}       {\widehat{\mathbf{x}}}
\newcommand{\hvy}       {\widehat{\mathbf{y}}}
\newcommand{\hvz}       {\widehat{\mathbf{z}}}

\newcommand{\vA}        {\mathbf{A}}
\newcommand{\vB}        {\mathbf{B}}
\newcommand{\vC}        {\mathbf{C}}
\newcommand{\vD}        {\mathbf{D}}
\newcommand{\vE}        {\mathbf{E}}
\newcommand{\vF}        {\mathbf{F}}
\newcommand{\vG}        {\mathbf{G}}
\newcommand{\vH}        {\mathbf{H}}
\newcommand{\vI}        {\mathbf{I}}
\newcommand{\vJ}        {\mathbf{J}}
\newcommand{\vK}        {\mathbf{K}}
\newcommand{\vL}        {\mathbf{L}}
\newcommand{\vM}        {\mathbf{M}}
\newcommand{\vN}        {\mathbf{N}}
\newcommand{\vO}        {\mathbf{O}}
\newcommand{\vP}        {\mathbf{P}}
\newcommand{\vQ}        {\mathbf{Q}}
\newcommand{\vR}        {\mathbf{R}}
\newcommand{\vS}        {\mathbf{S}}
\newcommand{\vT}        {\mathbf{T}}
\newcommand{\vU}        {\mathbf{U}}
\newcommand{\vV}        {\mathbf{V}}
\newcommand{\vW}        {\mathbf{W}}
\newcommand{\vX}        {\mathbf{X}}
\newcommand{\vY}        {\mathbf{Y}}
\newcommand{\vZ}        {\mathbf{Z}}

\newcommand{\ddx}       {\frac{\partial}{\partial x}}
\newcommand{\ddy}       {\frac{\partial}{\partial y}}
\newcommand{\ddz}       {\frac{\partial}{\partial z}}
\newcommand{\ddr}       {\frac{\partial}{\partial r}}
\newcommand{\ddt}       {\frac{\partial}{\partial\theta}}
\newcommand{\ddp}       {\frac{\partial}{\partial\phi}}

\newcommand{\al}        {\alpha}
\newcommand{\bt}        {\beta} 
\newcommand{\gm}        {\gamma}
\newcommand{\dl}        {\delta}
\newcommand{\ep}        {\epsilon}
\newcommand{\zt}        {\zeta}
\newcommand{\et}        {\eta}
\newcommand{\tht}       {\theta}
\newcommand{\io}        {\iota}
\newcommand{\kp}        {\kappa}
\newcommand{\lm}        {\lambda}
\newcommand{\ph}        {\phi}
\newcommand{\ch}        {\chi}
\newcommand{\ps}        {\psi}
\newcommand{\rh}        {\rho}
\newcommand{\sg}        {\sigma}
\newcommand{\om}        {\omega}

%%% Note: remember the macros \ddt and \ddp as well
\newcommand{\zen}       {\phi}   % zenith angle
\newcommand{\azi}       {\theta} % azimuth angle

\newcommand{\CH}[1]     {\left[\vphantom{\int}#1\right]}
\newcommand{\ov}        {\overline}

\renewcommand{\ss}      {\scriptstyle}

\newcommand{\EMPH}[1]{\emph{\RED{#1}}}
\newcommand{\DEFN}[1]{\emph{\PURPLE{#1}}}
\newcommand{\VEC}[1]    {\mathbf{#1}}

\newcommand{\ghost}{{\tiny $\color[rgb]{1,1,1}.$}}

\newcommand{\uc}{\uncover}

\newcommand{\bbox}[1]{
\[ \mbox{\begin{tikzpicture}%
   \draw(0,0) node[draw,thick,olivegreen,rectangle] {\color{black} #1};%
  \end{tikzpicture}} \]
}

\newcommand{\cbox}[1]{
\begin{center}\begin{tikzpicture}%
   \draw(0,0) node[draw,thick,olivegreen,rectangle] {\color{black} #1};%
\end{tikzpicture}\end{center}
}

% Foreground and background for tikz.
% This default is for screen mode.
\newcommand{\fg}{white}
\newcommand{\bg}{black}
\newcommand{\mg}{gray}


\begin{document}

\begin{center}
 \Large MAS243 - examinable formulae
\end{center}

\section*{Formula sheet}

You will get the following formula sheet in the exam:
\begin{displaymath}
\begin{aligned}
&
\cos (A\pm B) = \cos A \cos B \mp \sin A \sin B
\\
&
\sin (A \pm B) = \sin A \cos B \pm \cos A \sin B
\\
&
\tan (A \pm B) = \frac {\tan A \pm \tan B}{1 \mp \tan A \tan B}
\\
&
a \cos \theta + b \sin \theta = R \cos (\theta - \alpha )
{\mbox { where $R={\sqrt {a^{2}+b^{2}}}$ and
$\cos \alpha = \frac {a}{R}$, $\sin \alpha = \frac {b}{R}$}}
\\
&
\cos ^{2} \theta = \frac {1}{2} \left( \cos 2 \theta +1 \right)
\\
&
\cos ^{3} \theta = \frac {1}{4} \left( 3 \cos \theta +
\cos 3 \theta \right)
\\
&
\cos ^{4} \theta = \frac {1}{8} \left( 3 + 4 \cos 2 \theta
+ \cos 4 \theta \right)
\\
&
\sin ^{2} \theta = \frac {1}{2} \left( 1 - \cos 2 \theta \right)
\\
&
\sin ^{3} \theta = \frac {1}{4} \left( 3 \sin \theta -\sin 3 \theta
\right)
\\
&
\sin ^{4} \theta = \frac {1}{8} \left( 3 - 4\cos 2 \theta
+\cos 4 \theta \right)
\end{aligned}
\end{displaymath}

\section*{Formulae that you need to know}

\subsection*{Two-dimensional polar coordinates}

\begin{align*}
 x &= r\cos(\azi) & y &= r\sin(\azi) \\
 r &= \sqrt{x^2+y^2} & \azi &= \arctan(y/x) \\
 \ve_r &= (\cos(\azi),\;\sin(\azi)) & \ve_\tht &= (-\sin(\azi),\;\cos(\azi)) \\
 dA &= r\,dr\,d\tht
\end{align*}

\subsection*{Cylindrical polar coordinates}

\begin{align*}
 x &= r\cos(\azi) & y &= r\sin(\azi) \\
 r &= \sqrt{x^2+y^2} & \azi &= \arctan(y/x) \\
 \ve_r &= (\cos(\azi),\;\sin(\azi),\;0) &
 \ve_\tht &= (-\sin(\azi),\;\cos(\azi),\;0) & 
 \ve_z &= (0,0,1)=\vk \\
 dV &= r\,dr\,d\azi\,dz
\end{align*}

\subsection*{Spherical polar coordinates}

\begin{align*}
 x &= r\cos(\azi)\sin(\zen) &
   r &= \sqrt{x^2+y^2+z^2} \\
 y &= r\sin(\azi)\sin(\zen) &
   \ve_r &= (\cos(\azi)\sin(\zen),\;\sin(\azi)\sin(\zen),\;\cos(\zen)) \\
 z &= r\cos(\zen) &
  dV &= r^2\sin(\zen)\,dr\,d\azi\,d\zen
\end{align*}

\subsection*{Vector algebra}

For vectors $\va=(x,y,z)$ and $\vb=(u,v,w)$ we have

\begin{align*}
 \va . \vb &= xu+yv+zw & \text{(a scalar)} \\
 \va\tm\vb &= (yw-zv,\;zu-xw,\;xv-yu) 
  = \det\bsm \vi & \vj & \vk \\ x & y & z \\ u & v & w \esm 
   & \text{(a vector)} \\
 \|\va\| &= \sqrt{x^2+y^2+z^2} = \sqrt{\va.\va} 
   & \text{(a scalar)} \\
 \hva &= \frac{\va}{\|\va\|} 
    = \left(\frac{x}{\sqrt{x^2+y^2+z^2}},\;
            \frac{y}{\sqrt{x^2+y^2+z^2}},\;
            \frac{z}{\sqrt{x^2+y^2+z^2}}\right) 
   & \text{(a vector)}.
\end{align*}

Now let $\vn$ be another vector, and let $\va_{||}$ and $\va_\perp$ be
the components of $\va$ parallel and perpendicular to $\vn$.  If $\vn$
is a unit vector (ie $\|\vn\|=1$) then
\begin{align*}
 \va_{||} &= (\va.\vn)\vn &
 \va_\perp &= \va-\va_{||} = \va - (\va.\vn)\vn.
\end{align*}
If $\vn$ is not necessarily a unit vector, we have
\begin{align*}
 \va_{||} &= \frac{\va.\vn}{\vn.\vn}\vn = (\va.\hvn)\hvn &
 \va_\perp &= \va-\va_{||} = \va - \frac{\va.\vn}{\vn.\vn}\vn.
\end{align*}

\subsection*{Differential operators}

For a scalar field $f$ and a vector field $\vu=(p,q,r)$ we have
\begin{align*}
 \nabla(f) = \grad(f) &= (f_x,f_y,f_z)  & \text{(a vector field)} \\
 \nabla^2(f) &= f_{xx}+f_{yy}+f_{zz} & \text{(a scalar field)} \\
 \nabla.\vu = \dv(\vu) &= p_x+q_y+r_z  & \text{(a scalar field)} \\
 \nabla\tm\vu = \curl(\vu) &=
  (r_y-q_z,\;p_z-r_x,\;q_x-p_y)
   = \det \bbm \vi & \vj & \vk \\ \ddx & \ddy & \ddz \\ p & q & r \ebm
    & \text{(a vector field)} \\
 \\
 \nabla^2(\vu) &= (\nabla^2(p),\;\nabla^2(q),\;\nabla^2(r)) \\
  &= (p_{xx}+p_{yy}+p_{zz},\;q_{xx}+q_{yy}+q_{zz},\;r_{xx}+r_{yy}+r_{zz})
     & \text{(a vector field)} \\
\end{align*}

\subsection*{Geometry of surfaces}

You will need to be able to find the area element $dA$ and the
analogous vector quantity $d\vA$ for a variety of different
surfaces.  In some cases it is easiest to find $dA$ and the unit
normal vector $\vn$ separately and then $d\vA=\vn\,dA$.  In other
cases it is easiest to find $d\vA$ first and then $dA=\|d\vA\|$ and
$\vn=d\vA/\|d\vA\|$.  In some cases the formulae below will give the
inward normal when we want the outward normal, in which case we just
need to multiply by minus one.

\begin{itemize}
 \item[(a)] For a surface where the position vector $\vr=(x,y,z)$ is
  given in terms of two parameters $u$ and $v$, we have 
  \[ d\vA = \vr_u \tm \vr_v \;du\,dv =
      (x_u,y_u,z_u)\tm(x_v,y_v,z_v)\;du\,dv.
  \]
 \item[(b)] For a surface given in the form $z=f(x,y)$ we have
  \[ d\vA = (-f_x,-f_y,1) \,dx\,dy. \]
 \item[(c)] For the curved surface of a cylinder of radius $a$ centred
  on the $z$-axis, we have 
  \begin{align*}
   \vn &= (\cos(\azi),\sin(\azi),0) \\
   dA &= a\,dr\,dz 
  \end{align*}
 \item[(d)] For a sphere of radius $a$ centred at the origin, we have 
  \begin{align*}
   \vn &= (\cos(\azi)\sin(\zen),\;\sin(\azi)\sin(\zen),\;\cos(\zen))
   \\
   dA &= a^2\sin(\zen) d\azi\,d\zen 
  \end{align*}
\end{itemize}

\section*{Other formulae}

If you need any other formulae in the exam, then the exam question
will state the relevant formulae.

\end{document}
